%----------------------------------------------------------------------
%% start of file `main.tex'.
%% Copyright 2006-2013 Xavier Danaux <xdanaux@gmail.com>.
%% Edited 2018 Leonardo Mauro <leo.mauro.desenv@gmail.com>
% Available at https://www.overleaf.com/read/pxwngvmqqdcw
% Github: https://github.com/leomaurodesenv/curriculum-vitae

%----------------------------------------------------------------------
%-- main packages
%----------------------------------------------------------------------
% options: font size ('10pt', '11pt' and '12pt'), or font family ('sans' and 'roman')
\documentclass[10pt,a4paper,sans]{moderncv}
% styles: 'casual' (default), 'classic', 'oldstyle' and 'banking'
\moderncvstyle{classic} 
% colors: 'blue' (default), 'orange', 'green', 'red', 'purple', 'grey' and 'black'
\moderncvcolor{blue} 
% character encoding
\usepackage[utf8]{inputenc}
% adjust the page margins
\usepackage[scale=0.8, top=2cm, bottom=1cm]{geometry}

%----------------------------------------------------------------------
%-- required packages
%----------------------------------------------------------------------
\usepackage{import}
% options = {nopagenumber, titlemiddle}
\usepackage[titlemiddle]{moderncvplus}

%----------------------------------------------------------------------
%-- personal data
%----------------------------------------------------------------------
\name{Leonardo Mauro P.}{Moraes}
\title{Curriculum Vitae} % optional
%\address{address}{}{} % optional
\email{leo.mauro.desenv@gmail.com} % optional
\phone[mobile]{+55 16 99164-0477} % optional
\social[linkedin][www.linkedin.com/in/leomaurodesenv/]{leomaurodesenv}
\social[github][github.com/leomaurodesenv/]{leomaurodesenv}
\homepage{leomaurodesenv.github.io} % optional
%\extrainfo{additional information} % optional
%\photo[64pt][0.4pt]{photo.jpg} % optional
%\quote{Some quote} % optional

%----------------------------------------------------------------------
%-- content
%----------------------------------------------------------------------
\begin{document}
\makecvtitle

%----------------------------------------------------------------------
%-- resume
%----------------------------------------------------------------------
\vspace*{-1pt}
\small{As Principal MLOps Engineer, I use data to solve problems and improve customer experiences with Artificial Intelligence solutions. I have experience as a Data Scientist and Data Analyst, which helps me approach challenges from different angles. I also lead a team, focusing on mentoring and supporting new fellows.}
\vspace{0.5cm}

%----------------------------------------------------------------------
%-- education
%----------------------------------------------------------------------
\section{Education} \vspace{5pt}

\cventry{2022 - Present}{PhD in Computer Science}{University of Sao Paulo}{USP}{Brazil}{
\textit{Data Mining, Artificial Intelligence, Data Warehouse, Big Data.}}

\cventry{2017 - 2020}{Master in Computer Science}{University of Sao Paulo}{USP}{Brazil}{
\textit{Data Mining, Artificial Intelligence, Social Network, Time Series.}}

\cventry{2013 - 2017}{Bachelor of Computer Science}{Federal University of Mato Grosso do Sul}{UFMS}{Brazil}{\textit{Data Mining, Artificial Intelligence, Data Stream.}}

\cventry{2010 - 2011}{Technical Informatics for Internet}{ETEC Youth park}{ETEC}{Brazil}{\textit{Web Systems, Web Design.}}

%----------------------------------------------------------------------
%-- employment history
%----------------------------------------------------------------------
\section{Professional Experience} \vspace{5pt}

\cventry{Aug 2024 - Present}{Principal Machine Learning Engineer}{AB-InBev}{Brazil}{}{Product management, machine learning platform, and MLOps.}

\cventry{Sep 2020 - Present}{Data Science Tutor}{University of Sao Paulo}{Brazil}{}{Tutor of the Master of Business Administration (MBA) of Data Science.}

\cventry{Mar 2023 - Jul 2024}{Consultant Team Leader}{Amaris Consulting}{Brazil}{}{Machine learning, data science, development, and programming.}

\cventry{Nov 2020 - Mar 2023}{Machine Learning Engineer}{Sinch}{Brazil}{}{Data science, research, development, and programming.}

\cventry{Apr 2021 - Apr 2022}{Professor}{Fullture School}{Brazil}{}{Professor of Data Science course.}

\cventry{Ago 2019 - Oct 2020}{Data Analyst}{Argo Solutions}{Brazil}{}{Programming, data visualization, exploratory data analysis, and business intelligence.}

%----------------------------------------------------------------------
%-- skills
%----------------------------------------------------------------------
\section{Technical and Soft skills} \vspace{5pt}

\cvtext{
Check out an overview of my coding in  \href{https://profile.codersrank.io/user/leomaurodesenv/}{\texttt CodersRank}. \vspace{4pt}
\begin{itemize}
    \setlength\itemsep{3pt}
	\item \textbf{Programming:} Python, Spark and SQL.
	\item \textbf{Cloud computing:} Microsoft Azure, AWS and Google Cloud Platform.
	\item \textbf{General Business Skills:} Good presentation skills, works well in a team.
	\item \textbf{Languages:} Portuguese, English, and Brazilian Sign Language (LIBRAS).
	\item \textbf{Other:} Good research skills, can write well organized and structured reports.
\end{itemize}
}

%----------------------------------------------------------------------
%-- references
%----------------------------------------------------------------------
\section{References} \vspace{5pt}

\cvdoublecolumn{
\cvreference{Liz Zorzo}
	{Global Anti-fraud Manager}
    {Sinch}
    {Campinas, SP, Brazil}
    {}%email
    {liz-zorzo-19880525}%linkedin
}
{\cvreference{Alexandre Cordeiro}
    {Head of Product and Marketing}
    {Argo Solutions}
    {Sao Paulo, SP, Brazil}
    {}%email
    {alecordeiro}%linkedin
}

%----------------------------------------------------------------------
%-- publications
%----------------------------------------------------------------------

\newpage % force newpage

% Publications from a BibTeX file without multibib
%  for numerical labels: \renewcommand{\bibliographyitemlabel}{\@biblabel{\arabic{enumiv}}}% CONSIDER MERGING WITH PREAMBLE PART
%  to redefine the heading string ("Publications"): \renewcommand{\refname}{Articles}
\nocite{*}
\bibliographystyle{unsrt}
\bibliography{publications}

\vspace{0.5cm}
\cvtext{
Links:
\begin{itemize}
    \item \url{https://doi.org/10.1016/j.eswa.2024.125449}
    \item \url{https://doi.org/10.1007/978-3-031-64748-2_3}
    \item \url{https://doi.org/10.5753/dsw.2023.233602}
    \item \url{https://doi.org/10.5220/0011842700003467}
    \item \url{http://dx.doi.org/10.5220/0007728200930103}
    \item \url{https://repositorio.usp.br/item/002984641}
\end{itemize}}

%----------------------------------------------------------------------
%-- awards
%----------------------------------------------------------------------
\section{Awards} \vspace{5pt}

\cventry{Sep 2023}{Best Papers Selection}{SBBD Conference}{}{\url{https://sbbd.org.br/2023/}}{QASports: A Question Answering Dataset about Sports.}

\cventry{Apr 2023}{Best Papers Selection}{ICEIS Conference}{}{\url{https://iceis.scitevents.org/}}{BigQA: A Software Reference Architecture for Big Data Question Answering Systems.}

\cventry{Mar 2019}{2nd Place - SancaThon Future Farms}{EESC/USP}{}{\url{www.sel.eesc.usp.br/sancathon/}}{CaipiraBot: Intelligent system to weight chicken in a practice, simple and automatic way.}

\cventry{Jun 2018}{1st Place - SancaThon}{EESC/USP}{}{\url{http://www.sel.eesc.usp.br/sel/?p=5955}}{OurLife: System IoT for care the hospital patients; intelligent solution for Sao Carlos city.}

\cventry{Aug 2017}{JavaScript Innovation Award}{JS Classes}{}{\url{https://www.jsclasses.org/smm-maker-profile}}{Super Mario Maker is a platform that allows players to create and modify levels for Super Mario Bros game. This package can retrieve the player profile information from the Nintendo website.}

\cventry{Jul 2017}{1st Place - Hackathon}{44º SEMISH}{}{\url{http://csbc2017.mackenzie.br/eventos/44-semish}}{Daninha's Finder: Web system to find weeds in crop fields in IoT scenario.}

%----------------------------------------------------------------------
%-- Other types of Technical Production
%----------------------------------------------------------------------
\section{Patents} \vspace{5pt}

\cvtext{
\begin{itemize}
	\item TapAI (INPI). \textit{BR 51 2017 000379-3}. \vspace{3pt}
	\item SysCGenerator (INPI). \textit{BR 51 2017 000146-4}. \vspace{3pt}
	\item PassosEdu - Mat (INPI). \textit{BR 51 2017 000048-4}. \vspace{3pt}
	\item 2+ Dengue (INPI). \textit{BR 51 2016 001477-6}. 
\end{itemize}
}

%----------------------------------------------------------------------
\end{document}
