%----------------------------------------------------------------------
%% start of file `main.tex'.
%% Copyright 2006-2013 Xavier Danaux <xdanaux@gmail.com>.
%% Edited 2018 Leonardo Mauro <leo.mauro.desenv@gmail.com>
% Available at https://www.overleaf.com/read/pxwngvmqqdcw
% Github: https://github.com/leomaurodesenv/curriculum-vitae

%----------------------------------------------------------------------
%-- main packages
%----------------------------------------------------------------------
% options: font size ('10pt', '11pt' and '12pt'), or font family ('sans' and 'roman')
\documentclass[10pt,a4paper,sans]{moderncv}
% styles: 'casual' (default), 'classic', 'oldstyle' and 'banking'
\moderncvstyle{classic} 
% colors: 'blue' (default), 'orange', 'green', 'red', 'purple', 'grey' and 'black'
\moderncvcolor{black} 
% character encoding
\usepackage[utf8]{inputenc}
% adjust the page margins
\usepackage[scale=0.8, top=2cm, bottom=1cm]{geometry}

%----------------------------------------------------------------------
%-- required packages
%----------------------------------------------------------------------
\usepackage{import}
% options = {nopagenumber, titlemiddle}
\usepackage[titlemiddle]{moderncvplus}

%----------------------------------------------------------------------
%-- personal data
%----------------------------------------------------------------------
\name{Leonardo Mauro P.}{Moraes}
\title{Curriculum Vitae} % optional
%\address{address}{}{} % optional
\email{leo.mauro.desenv@gmail.com} % optional
\phone[mobile]{+55 16 99164-0477} % optional
\social[linkedin][https://www.linkedin.com/in/leonardo-mauro/]{leonardo-mauro}
\social[github][https://github.com/leomaurodesenv/]{leomaurodesenv}
\homepage{leonardomauro.com} % optional
%\extrainfo{additional information} % optional
%\photo[64pt][0.4pt]{photo.jpg} % optional
%\quote{Some quote} % optional

%----------------------------------------------------------------------
%-- content
%----------------------------------------------------------------------
\begin{document}
\makecvtitle

%----------------------------------------------------------------------
%-- resume
%----------------------------------------------------------------------
\vspace*{-6pt}
\small{I'm just another computer enthusiast. I am looking for the improvement in the two areas of which I am passionate: Web Systems and Artificial Intelligence. I seek to know new technologies and mechanisms because \textit{"the joy of thinking and learning makes us think and learn even more". (Aristotle)}}

%----------------------------------------------------------------------
%-- education
%----------------------------------------------------------------------
\section{Academic Qualifications} \vspace{5pt}

\cventry{2017 - 2019 \textit{(expected)}}{Master in Computer Sciences and Computational Mathematics}{University of Sao Paulo/Institute of Mathematics and Computer Sciences}{USP/ICMC}{Brazil}{
\textit{Data Mining, Artificial Intelligence, Social Network, Time Series.}}

\cventry{2013 - 2017}{Bachelor in Computer Science}{University Federal of Mato Grosso do Sul/Ponta Pora}{UFMS/CPPP}{Brazil}{\textit{Data Mining, Artificial Intelligence, Data Stream.}}

\cventry{2010 - 2011}{Technical in Informatics for Internet}{ETEC Park of the Youth}{ETEC}{Brazil}{\textit{Web Systems, Web Design.}}

%----------------------------------------------------------------------
%-- employment history
%----------------------------------------------------------------------
\section{Professional Experience} \vspace{5pt}

\cventry{Mar 2015 - Mar 2016}{Internship - Web Developer}{Inspectorate of the Brazilian Federal Revenue}{Brazil}{}{Analysis and development of web systems, database management, requirements analysis. I developed an Enterprise Resource Planning (ERP) web system to inventory control and activity monitoring.}

\cventry{Feb 2012 - Aug 2012}{Web Developer}{FastCash}{Brazil}{}{Development of web systems for intern operations, such as CRUD operations systems.}

%----------------------------------------------------------------------
%-- awards
%----------------------------------------------------------------------
\section{Awards} \vspace{5pt}

\cventry{Mar 2019}{2nd Place - SancaThon Future Farms}{EESC/USP}{}{\url{www.sel.eesc.usp.br/sancathon/}}{CaipiraBot: Intelligent system to weight chicken in a practice, simple and automatic way.}

\cventry{Jun 2018}{1st Place - SancaThon}{EESC/USP}{}{\url{http://www.sel.eesc.usp.br/sel/?p=5955}}{OurLife: System IoT for care the hospital patients; intelligent solution for Sao Carlos city.}

\cventry{Aug 2017}{JavaScript Innovation Award}{JS Classes}{}{\url{https://www.jsclasses.org/smm-maker-profile}}{Super Mario Maker is a Nintendo application that allows players to create and modify levels for the well known Super Mario Bros. game. This package can retrieve the profile information of a Super Mario Maker user from the Nintendo Web site using its API.}

\cventry{Jul 2017}{1st Place - Hackathon}{44º SEMISH}{}{\url{http://csbc2017.mackenzie.br/eventos/44-semish}}{Daninha's Finder: Web system to find weeds in crop fields in IoT scenario.}

\cventry{Mai 2017}{JavaScript Innovation Award}{JS Classes}{}{\url{https://www.jsclasses.org/tapai}}{Games are a way to entertain users that can also be used as means to study how people react to certain situations. This package implements a game that requires the users to interact. It records the user actions, so they can be analyzed later for further study.}


%----------------------------------------------------------------------
%-- skills
%----------------------------------------------------------------------
\section{Technical and Personal skills} \vspace{5pt}

\cvtext{
Check out an overview of my coding in \url{https://sourcerer.io/leomaurodesenv}.
\begin{itemize}
	\item \textbf{Programming Languages:} Proficient in: C++, Python, PHP, Javascript, Node.js, HTML5, CSS3, SQL. \textit{Cloud computing:} AWS, Azure, Oracle Cloud. \textit{Basic abilities:} Java, Arduino, Matlab.\vspace{2pt}
	\item \textbf{General Business Skills:} Good presentation skills, works well in a team. \vspace{2pt}
	\item \textbf{Languages:} Portuguese, English and Brazilian Sign Language (BSL).
	\item \textbf{Other:} Good research skills, can write well organized and structured reports.
\end{itemize}
}

%----------------------------------------------------------------------
%-- references
%----------------------------------------------------------------------
\section{References} \vspace{5pt}

\cvdoublecolumn{
% \cvreference{Robson Soares Silva}
% 	{Department of Computer Science}
%     {University Federal of Mato Grosso do Sul}
%     {Ponta Pora, MS, Brazil, 79907-414}
%     {robsonsoares.silva@gmail.com}
%     {}%tel
% }
\cvreference{Robson Leonardo Ferreira Cordeiro}
	{Department of Computer Science}
    {University of Sao Paulo}
    {Sao Carlos, SP, Brazil, 13566-590}
    {robson@icmc.usp.br}
    {}%tel
}
{\cvreference{Anderson Alves Camargo}
    {Fiscal Department}
    {Brazilian Federal Revenue}
    {Ponta Pora, MS, Brazil, 79900-000}
    {anderson.camargo@receita.fazenda.gov.br}
    {}%tel
}

%----------------------------------------------------------------------
%-- some projects
%----------------------------------------------------------------------
\newpage % force newpage
\section{Notable Projects} \vspace{5pt}

\cvdoublecolumn{
    \textbf{Tutorial Education Program (PET Frontier):} \textit{Activities of Education, Research and Extension, principally in robotic activities (2013 - 2017).}
    \small{Tutorial Education Program is a Brazilian governmental program, subsidized by Ministry of Education (MEC), which aims the high academic formation and professional excellence through extracurricular activities. The PET Frontier works mainly with robotic activities and competitions, such as Latin American Robotics Competition (LARC) and Brazilian Robotics Competition (CBR). I worked in some of these competitions elaborating programs and solutions, also instructed courses and speeches.}}{
	\textbf{Software Factory of UFMS/CPPP:} \textit{Development of web softwares and mobile educational softwares for Ponta Pora (2013 - 2015).}
	\small{The Software Factory of UFMS/CPPP aims the development of software to support the local education and progress, with applications that can be used by the local population, mainly in schools and universities. I worked on many web and mobile projects, which help me develop my technical and social skills. However, my projects are mainly web systems, in this way was produced four computational programs registered in the National Institute of Industrial Property (INPI).}}

%----------------------------------------------------------------------
%-- publications
%----------------------------------------------------------------------

% Publications from a BibTeX file without multibib
%  for numerical labels: \renewcommand{\bibliographyitemlabel}{\@biblabel{\arabic{enumiv}}}% CONSIDER MERGING WITH PREAMBLE PART
%  to redefine the heading string ("Publications"): \renewcommand{\refname}{Articles}
\nocite{*}
\bibliographystyle{unsrt}
\bibliography{publications}

%----------------------------------------------------------------------
%-- Other types of Technical Production
%----------------------------------------------------------------------
\section{Other Technical Production} \vspace{5pt}

\cvtext{
\begin{itemize}
	\item Leonardo Mauro Pereira Moraes. \textit{Arduino: Introduction to Basic Components.} 2017. \vspace{3pt}
	\item Leonardo Mauro Pereira Moraes. \textit{How to do: Brazilian Olympiad of Informatics (OBI).} 2016. 
\end{itemize}
}

%----------------------------------------------------------------------
\end{document}
